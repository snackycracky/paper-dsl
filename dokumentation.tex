

\documentclass[11pt,english,ngerman, headsepline]{scrreprt}
\usepackage{lmodern}
\renewcommand{\sfdefault}{lmss}
\renewcommand{\ttdefault}{lmtt}
\usepackage[T1]{fontenc}

\usepackage{listings}

\usepackage[utf8]{inputenc} \usepackage[a4paper]{geometry}
\geometry{verbose,tmargin=3cm,bmargin=3cm,lmargin=3cm,rmargin=2.75cm,headheight=1cm,headsep=0.666cm,footskip=1cm}
\setcounter{secnumdepth}{3}
\setcounter{tocdepth}{3}
\setlength{\parskip}{\medskipamount}
\setlength{\parindent}{0pt}
\usepackage{babel}
  \usepackage{tipa}
\usepackage{verbatim} 
\usepackage{float}  
\usepackage{courier}
\usepackage{url}
\usepackage{graphicx}
\usepackage{setspace}
\usepackage[square,sort,numbers]{natbib} \usepackage[utf8]{inputenc}
\usepackage{graphicx} 
\usepackage[xindy,toc]{glossaries}
 
 
 
\input{formating/formating.tex}

\parindent 0pt

\begin{document} 


\titlepage

\begin{center}
\includegraphics[width=5cm]{pics/FH-Logo}\vspace{0.5cm}

\par\end{center}

\noindent \begin{center}
\textsf{\textbf{\Large Fachbereich Informatik und Medien - \\
Wissenschaftliches Arbeiten und Schreiben - Master Inf. Prof. Loose
WS 2011/12}}\\ \textsf{\large }\\
\vspace{1cm}

\par\end{center}

\begin{center}
\textsf{\textbf{\huge Untersuchung des sprachorientierten
Programmierparadigmas Anhand von Metaprogrammierung und einer damit
erstellten DSL für Preispoitik in einem Boardinghaus.\\
}}\textsf{}\\ \textsf{}\\ 

\par\end{center}{\Large \par}

\vspace{2cm}


\noindent \begin{center}
{\huge }\begin{tabular}{rl}
Vorgelegt von: & Nils Petersohn B.Sc. \tabularnewline am: &
29.02.2012.\tabularnewline
\end{tabular}
\par\end{center}{\huge \par}

\vspace{1cm}

\begin{abstract}
 
Sprachorientierte Programmierung ist in dieser Arbeit betrachtet wurden.
Dazu wurde eine DSL mit Hilfe der Groovy-MOP erstellt, um daran das Paradigma
zusammen mit dem Domänenexperten unter folgenden Gesichtspunkten zu bewerten:
Lesbarkeit, intuitivem Verständnis und Flexibilität. 
Der Autor beschreibt intensiv die Erstellung der DSL und die
Metaprogrammierungs-Leistungsträger der gewählten Programmiersprache.

\end{abstract}
 

\noindent \begin{center}
\medskip{}
\begin{tabular}{rl}
\tabularnewline
\end{tabular}
\par\end{center}

\newpage{}

%
\begin{comment}
leere Seite nach dem Titelblatt

dann Aufgabenstellung
\end{comment}


\begin{comment}
%\pagestyle{scrheadings}    %Kopfzeile ein

\noindent \begin{center}
\textsf{\textbf{\large Selbstst\"andigkeitserkl\"arung}}
\par\end{center}{\large \par}

\noindent Hiermit erkl\"are ich, dass ich die vorliegende Arbeit zum
Thema

\smallskip{}


\noindent \begin{center}
\textsf{Medienverwaltung mit Eclipse und Equinox}
\par\end{center}

\smallskip{}


\noindent vollkommen selbstst\"andig verfasst und keine anderen als
die angegebenen Quellen und Hilfsmittel benutzt sowie Zitate kenntlich
gemacht habe. Die Arbeit wurde in dieser oder \"ahnlicher Form noch
keiner anderen Pr\"ufungsbeh\"orde vorgelegt.

\medskip{}


\noindent Brandenburg/Havel, den dd.MM.yyyy

\vspace{1.7cm}


\noindent Unterschrift 

\selectlanguage{ngerman}
\end{comment}
%\newpage{}

%\noindent \begin{center}
%\textsf{\textbf{\large Danksagung}}
%\par\end{center}{\large \par}


%\newpage{}
%\begin{abstract}

%\end{abstract}

%\selectlanguage{english}%
%\begin{abstract}
%This is a second abstract in english.

%\newpage{}
%\end{abstract}

\selectlanguage{ngerman}%
\tableofcontents{}

\pagestyle{scrheadings}    %Kopfzeile ein


% ==========================================================================
% DOCUMENT START
% ==========================================================================

\chapter{Einleitung} 

Die Sektionen würden bei der echten Arbeit wegfallen.

\section{Motivation (Heranführen an das Thema)}

Das Wort “Abstraktion” bezeichnet meist den induktiven Denkprozess des
Weglassens von Einzelheiten und des Überführens auf etwas Allgemeineres oder
Einfacheres [...] alsö[...] jenen Prozess, der Informationen söweit auf ihre
wesentlichen Eigenschaften herab setzt, dass sie psychisch überhaupt weiter
verarbeitet werden können. (nach \cite{wikiAbsraktion}) Die grundlegenden
Abstraktionsstufen in der Informatik sind wie folgt aufgeteilt: Die unterste
Ebene ohne Abstraktion ist die der elektronischen Schaltkreise, die elektrische
Signale erzeugen, kombinieren und speichern. Darauf aufbauend existiert die
Schaltungslogik. Die dritte Abstraktionsschicht ist die der Rechnerarchitektur.
Danach kommt eine der obersten Abstraktionsschichten: “Die Sicht des
Programmierers”, der den Rechner nur noch als Gerät ansieht, das Daten speichert
und Befehle ausführt, dessen technischen Aufbau er aber nicht mehr im Einzelnen
zu berücksichtigen braucht. (nach S. 67 \cite{rechenberg2000informatik}). Diese
Beschreibung von Abstraktion lässt sich auch auf Programmiersprachen übertragen.
Nur wenige programmieren heute direkt Maschinencode, weil die
Programmiersprachen der dritten Generation (3GL) soviel Abstraktionsgrad bieten,
dass zwar kein bestimmtes Problem aber dessen Lösung beschrieben werden kann.
Die Lösung des Problems muss genau in der Sprache beschrieben werden und setzt
das Verständnis und die Erfahrung in der Programmiersprache und deren
Eigenheiten zur Problemlösung voraus. Das Verständnis des eigentlichen Problems,
dass es mit Hilfe von Software zu lösen gilt, liegt nicht immer zu 100\% bei dem
Programmierer, der es mit Java oder C\# bzw. einer 3GL lösen soll. Komplexe
Probleme z.B. in der Medizin, der Architektur oder im Versicherungswesen sind
oft söumfangreich, dass die Aufgabe des “Requirements Engeneering”
hauptsächlich darin besteht, zwischen dem Auftragnehmer und Auftraggeber eine
Verständnisbrücke zu bauen. Diese Brücke ist auf der einen Seite mit
Implementierungsdetails belastet und auf der anderen mit domänenspezifischem
Wissen (Fach- oder Expertenwissen). Die Kommunikation der beiden Seiten kann
langfristig durch eine DSL begünstigt werden, da eine Abstrahierung des
domänenspezifischen Problems angestrebt wird. Die Isolation der eigentlichen
Businesslogik und eine intuitiv verständliche Darstellung in textueller Form
kann sogar soweit gehen, dass der Domänenexperte die Logik in hohem Maße selbst
implementieren kann, weil er nicht mit den Implementierungsdetails derselben und
den syntaktischen Gegebenheiten einer turingvollständigen
General-Purpose-Language wie Java oder C\# abgelenkt wird. Im Idealfall kann er
die gewünschten Anforderungen besser abbilden.(vlg. \cite{heiseMPS2}). Beispiele
für DSL sind z.B.: Musiknoten auf einem Notenblatt, Morsecode oder
Schachfigurbewegungen (“Bauer e2-e4”) bis hin zu folgendem Satz: “wenn (Kunde
Vorzugsstatus hat) und (Bestellung Anzahl ist größer als 1000 oder Bestellung
ist neu) dann ... ”. “Eine domänenspezifische Sprache ist nichts anderes als
eine Programmiersprache für eine Domäne. Sie besteht im Kern aus einem
Metamodell einer abstrakten Syntax, Constraints (Statischer Semantik) und einer
konkreten Syntax. Eine festgelegte Semantik weist den Sprachelementen eine
Bedeutung zu.” (vlg. S.30 \cite{mdaDPunkt}). 


\section{Begriffserklärung}
Eine DSL beinhaltet ein Metamodell. In einer Domänengrammatik gibt es das
Konzept an sich, das beschrieben werden soll. Konzepte können Daten, Strukturen
oder Anweisungen bzw. Verhalten und Bedingungen sein. Das Metamodell oder auch
das Semantische Modell besteht aus einem Netzwerk vieler Konzepte. Das Paradigma
“Language Orientated Programming” (LOP) identifiziert ein Vorgehen in der
Programmierung, bei dem ein Problem nicht mit einer GPL (general purpose
language) angegangen wird, sondern bei dem zuerst domänenspezifische Sprachen
entworfen werden, um dann durch diese das Problem zu lösen. Zu diesem Paradigma
gehört auch die Entwicklung von domänenorientierten Sprachen und intuitive
Programmierung (intentional programming). “Intentional Programmierung ist ein
Programmierparadigma. Sie bezeichnet den Ansatz, vom herkömmlichen Quelltext als
alleinige Spezifikation eines Programms abzurücken, um die Intentionen des
Programmierers durch eine Vielfalt von jeweils geeigneten
Spezifikationsmöglichkeiten in besserer Weise auszudrücken.” (vlg.
\cite{wikiIntentional}) Es werden zwei Arten von DSLs unterschieden
\cite{fowler2011domain}. 


\section{Aufgabenstellung (kurz, knapp, präzise) und Erwartungen}

Das Paradigma der sprachorientierten Programmierung und die damit verbundenen
Konzepte der domänenorientierten Programmierung sollen in dieser Arbeit
analysiert und geprüft werden. 


Eine einfache DSLs soll mit Hilfe von
Groovy-Metaprogramming als interne DSLs entworfen werden.

Dabei stellen sich folgende Fragen: Wie effizient ist die Erstellung einer DSL
mittels der Programmiersprache Groovy und wie reagiert der Domänenexperte auf
die DSL?

Der Domänenexperte, die keine Erfahrung mit Programmierung haben, sollen an
interne DSLs für deren bekannte Domäne herangeführt werden, um diese
anschließend nach Lesbarkeit, intuitivem Verständnis und Flexibilität zu
bewerten. Dabei bekommt der Proband mehrere Aufgaben, die mit den
gegebenen DSLs zu lösen sind.
 
 %TODO erwartungen?
 
\section{Gliederung}

Zuerst werden theoretische Grundlagen zum Thema “Sprachorientierte
Programmierung” und Metamodellierung erläutert. Der praktische Teil beginnt mit
der Vorstellung der Problemdomäne und deren verschiedenen Anforderungen. 
Der nächste Schritt in dieser Arbeit beschäftigt sich mit der konzptionellen
Implementierung der DSL.  Weiterhin
soll die Testumgebung mit den Probanden spezifiziert werden, um danach die
Durchführung und die Ergebnisse auszuwerten und zu beschreiben.

\section{Abgrenzung}
Der Fokus dieser Arbeit soll auf die textuelle und nicht auf die graphische
Repräsentation von DSLs abzielen. “Natural language processing” soll nur
oberflächlich betrachtet werden. Der praktische Teil berichtet ausschließlich
über die Groovy-Metaprogrammierung. 
%TODO nu interne DSLs werden praktisch angewendet.s und kein MPS

\chapter{Theorie}

\section{Domain Specific Languages}
 

Eine domänenspezifische Sprache (engl. domain-specific language, DSL) ist, im
Gegensatz zu gängigen Programmiersprachen, auf ein ausgewähltes
 Problemfeld (die Domäne) spezialisiert. Sie besitzt hoch spezialisierte
Sprachelemente mit meist natürlichen Begriffe aus der Anwendungsdomäne.
Das Gegenteil einer domänenspezifischen Sprache ist eine universell einsetzbare
Programmiersprache (engl. general purpose language, GPL), wie C und Java, oder
eine universell einsetzbare Modellierungssprache, wie UML.

Mit Hilfe einer solchen Sprache können ausschliesslich Problembeschreibungen
innerhalb des jeweiligen Problemgebiets beschrieben werden.
Andere Problembereiche sollen ausgeblendet werden, damit der Domänenexperte sich
nur auf das für Ihn wichtigste in dem jeweiligen Bereich konzentrieren kann.

Der Domänenspezialist (z.B. ein Betriebswirt) ist mit dem Problembereich (z.B.
Produktpreisbildung) sehr vertraut. Die Domänensprache, z.B. zur Beschreibung
von Preisbildungskomponenten und deren Zusammenhänge, gibt dem Betriebswirt ein
mögliches Werkzeug, um die Preise für Produkte (z.B. Computerhardware) dynamisch
anzupassen. Diese DSL ist dann aber für andere Bereiche, wie z.B.
der Aufstellung des Personalschichtplans nicht einsetzbar.

Die Charakteristiken einer DSL sind vorzugsweise minimale Syntax die nur die
nötigsten Mittel zur Strukturierung benötigt um die Lesbarkeit zu erhöhen und
keine Ablenkung vom Problem zu schaffen. 
Was genau eine minimale Syntax ausmacht ist schwer messbar zu machen.
Vorzugsweise sollte die Syntax keine Redundanzen aufweisen, wie das z.B. bei
XML der Fall ist indem das offene und geschlossen element nochmals den selben
namen tragen. Als Faustregel sollte ein Zeichen bzw. eine minimale Kombination
aus mehreren Zeichen und deren Position in der DSL für eine Informationseinheit
verwendet werden. Als Zeichen sind hier insbesondere leerzeichen und
Zeilenumbrüche gemeint. Solche die auch eine angelehnte Bedeutung zu der
natürlichen Sprache haben. Z.B. Klammern, Semikolen, Kommas, Punkte, Slashes \ldots 

Fowler ist der begründeten Ansicht, dass zu einer DSL immer ein semantisches
Modell \cite[p. 159]{fowler2011domain} exsistieren sollte, das unabhängig von
der eigentlichen DSL ist aber direkt dazugehört. Es ist das Modell zu der DSL 
oder auch das Schema. Die DSL instanziiert das semantische Modell und ist damit
eine gut lesbare Form der Modellinstanziierung. Das semantische Modell hat somit
große Ähnlichkeit mir einem Domänen Modell \cite[]{fowler2003patterns}.
Vorteilhaft ist damit die klare Trennung von Angelegenheiten
\cite{Hürsch95separationof}, auf der einen Seite das Parsen der DSL durch den
Parser und auf der anderen Seite die daraus resultierenden Semantiken.
Die DSL ist damit eine Abbildung des Modells. Veränderungen können an diesem
formalen Modell separat zur DSL durchgeführt werden. Z.B. kann das semantische
Model das von einer Zustandsmaschine sein wie in Abbildung
\ref{state-machine-semantic-model} graphisch dargestellt.


\begin{figure}[h!]
	\begin{center}
	\includegraphics[width=0.7\textwidth]{pics/data-model.jpg}
	\end{center}
	\caption{Semantisches Modell einer 	Zustandsmaschine
	- neilvandyke.org/smores/ }
	\label{state-machine-semantic-model}
\end{figure}

\begin{figure}[h!]
	\begin{center}
	\includegraphics[width=0.7\textwidth]{pics/zutandsdiaGeldautomat.jpg}
	\end{center}
	\caption{Zustandsdiagramm Geldautomat
	\url{https://www.fbi.h-da.de/uploads/RTEmagicC_f2da95d8df.gif.gif} }
	\label{zutandsdiaGeldautomat.jpg}
\end{figure}

z.B. ist das Metamodell von Zuständen eines Geldautomaten das Modell einer
Zustandsmaschine (Abb. \ref{state-machine-semantic-model}). Das Zustandsdiagramm
des Geldautomaten ist damit das in Abbilung \ref{zutandsdiaGeldautomat.jpg}
dargestellte. Es beschreibt alle Zustaende und Zustandsübergänge bzw. deren
vorausgehende Ereignisse bzw. Aktionen. Eine DSL könnte dieses Semantische
Modell also Metamodell der Zustandsmaschine mit den Gegebenheiten des
Geldautomaten instanziieren. (Code Listing \ref{geldautomat.dsl}).
   
\lstinputlisting[caption={DSL Ausschnitt für ein Geldautomat},
label={geldautomat.dsl},style=default]{code/geldautomat.dsl}
 
Im Sprachsektor des Techonologieradars (Juli 2011 von
Thoughtworks)\footnote{http://www.thoughtworks.com/radar}
sind sind die domänenspezifischen Sprachen unverändert nahe dem Zentrum
angesiedelt. Thoughtworks ist der Meinung, dass DSLs eine alte Technologie ist
und bei Entwicklern einen unterschätzten Stellenwert
hat. (nach \cite{thoughtworks-tr}) Diese Quelle steht aber unter Vorbehalt in
Hinblick auf den Verkauf des Buches von Martin Fowler (Chief Scientist von
Thoughtworks) und Rebecca Parsons (CTO von Thoughtworks) über DSLs\footnote{Domain Specific
Languages, Addison Wesley 2011} 


\subsection{Unterscheidungen}

Martin Fowler unterscheidet zwischen ausprägungen solcher DSLs, indem er deren
Beziehung zu einer GPL benennt. Externe DSLs sind eigenständige und unabhängige
Sprachen die einen eigenen sepziell angefertigten Parser besitzen. 

``Sowohl die konkrete Syntax als auch die Semantik können frei definiert
werden. SQL oder reguläre Ausdrücke sind Vertreter von externen DSLs. Wenn eine
DSL innerhalb bzw.'' \cite{wikidsl}
mit einer GPL definiert wurde, nennt er diese interne DSL. Solche eingebettete
Spracherweiterungen sind mit den gegebenene Mittel der ``Wirtssprache'', oft deren Möglichkeit zur
Metaprogrammierung (Kapitel \ref{metaprogrammingLabel}), erstellt.
Vorzugsweise sind solche Wirtssprachen dynamisch typisiert wie z.B. Ruby, Groovy
oder Scala. 
``Dadurch sinkt der Implementierungsaufwand. Eine interne DSL ist immer eine echte
Untermenge einer generelleren Sprache.'' \cite{wikidsl}

DSLs die nicht in der Hostsprache implementiert sind werden trotzdem von einer
anderen Programmiersprache geparst und weiterverarbeitet. Zusätzlich ist man bei
einer separaten Sprachdefinition nicht syntaktisch eingeschränkt. Damit sind vor
allem Szenarien gemeint bei denen z.B. Methodennamen hinter die
Argumentenklammer geschrieben ist: ``(1,1)addiere'' oder sonstige nur denkbare
syntaktische Abwandlugnen.


 
% TODO vor und nachteile des jeweiligen typs.

\section{Sprachorientierte Programmierung}

Ein Programmierer einer Anwendung auf der Internet-Platform benutzt viele
Sprachen die alle für einen bestimmten Zweck eingesetzt werden. HTML für die
Struktur und CSS für die visuelle Repräsentation dieser Stuktur und SQL um Daten
aus der Datenbank abzufragen bzw. aufzuwerten. Der Entwickler muss sich dann für
ein Web-Framework entscheiden, das in einer GPL wie
Java, Ruby oder PHP geschrieben ist. Nicht zuletzt ist Javascript noch eine
zusätzliche Sprache die Client oder auch Serverseitig\footnote{nodejs.org}
eingesetzt wird. 
Das Paradigma sprachorientierte Programmierung ermuntert den Programmier dazu
eigene Sprachen zu entwerfen die den jeweiligen Problembereich explizit
beschreiben.

``Language oriented programming (LOP) is about describing a system through
multiple DSLs[\ldots]LOP is a style of development which operates about
the idea of building software around a set of DSLs''.(vlg.
\cite{fowler2005language})

Sprachorientierte Programmierung ist nach Fowlers Beschreibung als ein Vorgehen
bei dem eine Programmiersprache sich auch mehreren Unter-Programmiersprachen
zusammensetzt, deren Funktionen mit definierten Sprachmodellen beschrieben
werden.
Diese Sprachmodelle sollen die fachlichen Probleme auf eine
natürliche Art beschreiben können, ohne dabei mehr zu beschreiben als das
Fachgebiet benötigt. Dabei liegt der Fokus auf der Problembeschreibung und nicht
auf der Lösungsbeschreibung.

In dem oft zitierten Paper beschreibt Fowler ein kleines Programm, das dazu
dient Textdateien, die eine bestimmte Struktur haben in Objektinstanzen zu
überführen.
Eine EDI-Nachricht \footnote{http://en.wikipedia.org/wiki/EDIFACT} (Electronic
Data Interchange) ist ein Beispiel dafür (Listing \ref{edimsg.txt}).

\lstinputlisting[caption={EDIFACT Nachricht für einen Verfügbarkeitsanfrage},
label={edimsg.txt},style=default]{code/edimsg.txt}

Je nachdem wie z.B. die ersten drei Zeichen einer Zeile sind, wird beim
Parser eine bestimmte ``Parsing-Strategie'' angewedet. \cite{gamma1995design}
Die angewendete Strategie wird dann dazu verwendet um den Rest der Zeile
auszulesen. Diese Strategie kann zusätzlich konfiguriert werden. 
Folgendes ist nun konfigurierbar: Die ersten drei Zeichen an sich. Weithin die
Zielklasse die je nach den ersten drei Zeichen mit den restlichen Daten der
Zeile als Klassenvariablenwerte instanziiert werden soll und vor allem an
welcher Stelle der Zeile der Wert einer bestimmten Klassenvariable vorkommt.

Mit diesem kleinen Programm wurde eine Abstraktion gebaut, die durch
Konfiguration der Strategien spezifiziert werden kann.
Also um die Abstraktion zu benutzen müssen die Strategien konfiguriert werden
und deren Instanzen an den Reader-Treiber übergeben werden  (Listing
\ref{readerstrategyAndDriver}).
 

\lstinputlisting[caption={Instanzen der verschiedenen Strategien},
label={readerstrategyAndDriver},style=default]{code/readerstrategyAndDriver.java}

Da diese Konfigurationen besser konfigurierbar machen ohne immer neuen Bytecode
genierieren zu müssen könnte man eine
YAML\footnote{http://de.wikipedia.org/wiki/YAML}-Datei schreiben und diese als
Input für die Strategiekonfiguration benutzen  (Listing
\ref{readerstrategyAndDriverYAML.txt}).
 
\lstinputlisting[caption={Instanzen der verschiedenen Strategien - YAML},
label={readerstrategyAndDriverYAML.txt},style=default]{code/readerstrategyAndDriverYAML.txt}

Jemand, der den Parser und dessen Anwendung versteht, kann in kurzer Zeit etwas
mit der YAML Datei anfangen. Genau der Inhalt dieser YAML-Datei ist nun schon
eine kleine Sprache. Die konkrete Syntax ist genau das YAML Format. Eine andere
konkrete Syntax könnte das XML Format sein. Da dieses aber zu ``verbos'' ist
dient es nicht der leserlichkeit. Die Abstrakte Syntax ist nun die
Basisstruktur: ``Mehrere Abbildungen von Zeilentypen auf Klassen. Jeweils mit
den drei Buchstaben, der Zielklasse und deren Felder bzw. deren Position''.
Egal ob in XML oder in YAML, die abstrakte Syntax bleibt immer gleich.

Wenn eine minimalistische Syntax eine Vorraussetzung für eine DSL ist, kann man
die YAML-Datei auch in Ruby darstellen (Listing
\ref{readerstrategyAndDriverRuby}). Das hat zur Folge, dass der Inhalt der DSL
mit einem Ruby Interpreter gelesen und verarbeitet werden kann. Wenn auch der andere Code in Ruby geschrieben sein wuerde (Stragie
Implementation, AddFieldExtractor, AddStrategy, \ldots) dann wäre Code Listing
\ref{readerstrategyAndDriverRuby} eine interne DSL und Ruby die Wirtssprache.
Also ein Untermenge von Ruby und eine konkrete- zu unseren abstrakten
Syntax.


\lstinputlisting[caption={Instanzen der verschiedenen Strategien - RUBY},
label={readerstrategyAndDriverRuby},style=default]{code/readerstrategyAndDriverRuby.txt}

%TODO language workbenches vorstellen



\section{Metaprogrammierung}\label{metaprogrammingLabel}

Mittels Metaprogrammierung ist ein Programmierer in der Lage Quelltext zur
Laufzeit zu ändern. Ein Meta-Objekt-Protokoll (MOP) stellt diese dafür
benötigten Mittel bereit.



\section{Groovy und Eigenschaften für eine interne DSL}

Diese Programmiersprache wurde entwurfen, um auf der JVM (Java Virtual Machine)
ausgeführt zu werden. Ruby, Python, Dylan und Smalltalk dienten als Inspiration
für die Entwickler von Groovy.

Eine Maxime für den Entwurf von Groovy war die hochgradige Kompatibilität zu
Java. Die Sprache ist auch syntaktisch stark an Java angelehnt. Wenn eine .java
Datei in eine .groovy Datei umbenannt wird, dann ist diese genau so ausführbar.
Groovy Klassen können auch von Java Klassen erben.

Groovy besitzt Eigenschaften, die sich besser als die von Java eignen, um eine
DSL zu entwerfen. Dazu zählt die wahlweise dynamische Typisierung, Closures,
native Syntax für Maps, Listen und Reguläre Ausdrücke, ein einfaches
Templatesystem, eine
XQuery-ähnliche Syntax zum Ablaufen von Objektbäumen, Operatorüberladung und
eine native Darstellung für BigDecimal und BigInteger.  

In den nächsten Abschnitten handelt diese Arbeit von den wichtigsten Features
dieser Programmiersprache, die sich für die Erstellung einer internen DSL
eignen. 


\section{MOP Leistungsträger}
Das MOP besteht aus den Haupt-Leistungsträgern: ``Metaclass'',
``Kategorien'' und ``Expandos''.

\subsection{Expando}\label{expando}

Expandos sind dynamische repräsentationen von Groovy-Beans. Deren
Klassenvariablen sind über get und set erreichbar. Wenn eine Variable von einem
Expando angefragt wird und diese nicht exsistiert, dann wird
keine Exception geworfen sondern null zurückgegeben. Auch wenn eine Variable
belegt wird, dann wird diese im Expando erzeugt, wenn diese noch nicht
exsistiert. Es ist auch möglich eine Colsure einer Klassenvariable zuzuweisen.
Diese ist dann genau wie eine Methode aufrufbar. Die Notation ist ``def x =
new Expando()''

\subsection{Kategorien}

Wenn eine Methode einer bereits exsistierenden Klasse zur Laufzeit hinzugefügt
werden soll, dann erfolgt das mittels ``Kategorien''.
Um eine Methode x zu der vorhandenen Klasse java.lang.Number hinzuzufügen ist
das erstellen einer neuen Klasse mit einer statischen Methode x erforderlich.
Das erste Argument der Methode ist dann eine Referenz auf die Instanz.
Durch die Notation \texttt{use(KlasseMitStatischemX)\{x kann jetzt auf eine
Instanz von Number angewendet werden.\}}


 \subsection{Metaclass}\label{Metaclass}
 
Jede Groovy Klasse implementiert die ``groovy.lang.GroovyObject''
 Schnittstelle.
Dadurch ist die Methode \texttt{getMetaClass} auf jedem Groovy-Objekt
vorhanden.
Die \texttt{MetaClass} Klasse wird dadurch referenziert.
Bei verwendeten Java-Klassen ist kein solches Interface aber die
\texttt{MetaClass} wird trotzdem jenen zugewiesen und in einer
\texttt{MetaClass} Registrierung \cite{fowler2003patterns}
verwaltet.
In der \texttt{MetaClass} werden alle Metainformationen der eigentlichen Klasse
verwaltet. Die \texttt{MetaClass} erlaubt es, nach dem Expando (Abschnitt
\ref{expando}) Prinzip, neue Klassenvariablen und Methoden einem Objekt
hinzuzufügen.
Javaklassen beschreiben das Verhalten zur Zeit der Übersetztung, Metaklassen
das zur Ausführungszeit.
Die Kardinaliät zwischen Objekt und Metaklasse ist immer 1:1.
Die \texttt{MetaClass} implementiert die Methode \texttt{invokeMethod}

\subsection{Closures}

Closures sind abgeschlossene Code-Abschnitte, die an sich einen eigenen
Datentyp darstellen. Closures werden von geschweiften Klammern abgegrenzt.
Innerhalb einer Closure sind Paramenter bzw. Argumente vom Ausführungscode durch
einen Pfeil getrennt \texttt{->}. Parameter vor dem Pfeil sind durch Kommas
getrennt.
Closures bestitzen das keywort \texttt{this} aber auch noch \texttt{owner} und
\texttt{delegate}

\paragraph{owner}
Dem Attribut Owner ist eine Referenz auf die Elternklasse. Diese Eigenschaft
kann nicht geändert werden. \texttt{Owner} ist in den meisten Fällen gleich der
Referenz von  \texttt{this}, außer wenn die umschließende Closure in einer
andern Closure liegt.
\paragraph{delegate}
Diese Eigenschaft ist genau wie die \texttt{owner} Eigenschaft zu verstehen,
jedoch mit dem bedeutenden Unterschied, dass dem \texttt{delegate} ein anderes
Objekt zugeordnet werden kann. Das ist mit der \texttt{owner} Eigenschaft nicht
möglich. So kann z.B. eine Closure einem fremden Objekt zugeordnet werden,
jedoch ist das Elternobjekt immer noch das in dem die Closure definiert wurde.
Mit der Neuzuweisung von delegate zum Fremdobjekt ist die Elternreferenz
veränderbar.

\subsection{ExpandoMetaClass}

Die Metaclass Eigenschaft aus Abschnitt \ref{Metaclass} ist ein Expando
(Abschnitt \ref{expando}).
Diese ExpandoMetaClass kann auch dazu verwendet werden um z.B. Methoden zu
überschreiben. Eine Statische Methode kann mittels folgendem Ausdruck hizugefügt
oder überschrieben werden: \texttt{Klasse.metaClass.static.x = \{
Closure-Inhalt\}}. 
Methoden können auch überladen werden. Konstruktoren können hizugefügt oder
überschrieben werden. 
  


\subsection{groovy syntaxeigentschaften}
Omitting parentheses
Groovy allows you to omit the parentheses for top-level expressions, like with the println command:

println "Hello"
method a, b
vs:

println("Hello")
method(a, b)
When a closure is the last parameter of a method call, like when using Groovy's 'each‘ iteration mechanism, you can put the closure outside the closing parens, and even omit the parentheses:

list.each( { println it } )
list.each(){ println it }
list.each  { println it }
Always prefer the third form, which is more natural, as an empty pair of parentheses is just useless syntactical noise!

There are some cases where Groovy doesn't allow you to remove parentheses. As I said, top-level expressions can omit them, but for nested method calls or on the right-hand side of an assignment, you can't omit them there.

def foo(n) { n }

println foo 1 // won't work
def m = foo 1
 

\chapter{MDA / MDSD und DSM Unterschiede} 
 
 
\chapter{Praktischer Teil}

Im praktischen Teil soll beschrieben werden, wie eine interne DSL mit Hilfe der
Groovy-Metaprogrammierung erstellt wurde und wie der Domänenexporte darauf
reagiert hat. 

\section{Die Fachliche Domäne}
Der fachliche Bereich im gesamten Kontext ist die Hotellerie. Diese
Arbeit betrachtet die Betriebswirtschaftliche Unterdomäne
 und darin, noch spezieller die tagesabhängige Preisbildung für Hotelzimmer.

\section{Vorgehen}
Nach der Zieldefinierung soll eine Bestandsaufnahme gemacht werden um die
Rahmenbedingungen für das Experiment offenzulegen. Danach werden die
Vorüberlegungen zur Zielerreichung dargestellt und anschließend die
Implementierung der Lösung beschrieben. Daraus entstehen jeweils Unterziele die
in den einzelenen Abschnitten näher beschrieben sind. Anschliessend bietet diese
Arbeit jeweils eine Ausgewertung und Beurteilung.

\section{Zieldefinierung}
Das Ziel war es eine Sprache zu erstellen, die sich ausschließlich durch
Preisbestimmung für jeden möglichen Tag in der Zukunft bzw. für jedes Apartment
im Hotel definiert.

\section{Bestandsaufnahme}
Zuerst wurde unverbindlich eine Bestandsaufnahme gemacht. Das Hotel
``Schoenhouse Apartments'' besteht aus 50 Apartments in Berlin Mitte. Der
Geschäftsführer ist Dipl.-Ing. Immanuel Lutz (Domänenexperte). Dieser bestimmt
auch hauptsächlich die Preisbildung der Apartments. Weiterhin besitzt das Hotel ein in Java
geschriebenes Property-Management-System (PMS), das zur Verwaltung folgender
Hauptkomponenten dient: Zimmer-, Gäste-, Apartment-, Sonderleitungs- und
Preisverwaltung.
Derzeit wird ein neuartiges PMS erstellt, dass auf Groovy-und-Grails basiert.
 
 \section{Vorüberlegungen}

Die Vorüberlegung erfolgte ohne den Domänenexperten lediglich die Zustimmung
für das Experiment ``Textuelle-Preisberechnung'' war gegeben, Da der
Domänenexperte nur wenig Zeit dafür preisgeben wollte. Die Vorüberlegungen
bestanden hauptsächlich aus der Grammatikstruktur und deren Semantik. 

Die DSL soll die Geschäftslogik für die dynamische Bildung der Zimmerpreise
beschreiben. In einem Hotel sind die Preise abhängig von Faktoren wie
``Angebot und Nachfrage auf dem Markt'', Investitionskosten, Zimmerkategorie,
Nebenkosten, Rabattaktionen, Provisionen der Geschäftspartner für eine
Zimmervermittlung, Zeitraum und überschneidende Ereignisse in der Umgebung
\cite[S. 44]{hahn1993tourismuspsychologie}. Mit Ereignissen sind Veranstaltungen
oder Feiertage sowie Saisons gemeint, die die Angebot und Nachfragen beeinflussen. Bewertungen, die
das Hotel auf Buchungsportalen bekommen hat sind auch Preisentscheident. Wenn
z.B. mehrere schlechte Bewertungen abgegeben wurden und darauf nur noch wenige
Gäste Buchen muss überlegt werden, ob das mit dem Preis zu regulieren geht.
Nicht zuletzt beeinflusst die Auslastung einer Zimmerkategorie oder die
Gesamtauslastung des Hotels den Zimmerpreis. Das bedeutet, wenn nur noch ein
Zimmer im Hotel verfügbar ist, dann kann es entsprechend teuer verkauft werden.
Die Auslastung, Ereignise und die Tage bis zu den Ereignissen zusammen
kombiniert beinflussen den Preis weiter. Auch die aktuelle Liquidität des
Unternehmens kann einfluss darauf haben. Der Zimmerpreis ist auch sensibel
gegenüber den Preisen der direkten Konkurenz in der Umgebung.
Der Faktor Markt ist der wohl am schwersten zu determinierende, da er sich aus
vielen anderen Faktoren zusammensetzt. Dazu gehört z.B. die Beziehung zwischen
angebotenen Hotelzimmern und nachgefragten. Wenn die Auslastung steigt und die
Nachfrage gleich bleibt, dann resultiert das in steigende Preise.
Bei sozialen, kulturellen oder positischen ereignissen weichen die Zimmerpreise
erheblich von der ``Rac-Rate'' (Grundpreisrate) ab.
Es stellt sich als schweirig heraus, alle Faktoren deterministisch zu
modellieren, da vor allem der subjektive Geschmack oder persönliche Motivationen
der potentiellen Gäste nur über statistische Werte berechenbar sind. Genau so
ist es mit der Faktorenauswahl bei ökonomischen bzw. volkswirtschaftlichen
Werten, um die Kaufkraft der internationalen Gäste zu bestimmen.
Formal kann mit diskreten Werten modelliert werden, die in direkter Beziehung zu
dem Hotel stehen. Indirekte Beziehungen werden hier aus den oben genannten
Gründen nicht betrachtet.

Der Geschäftsführer muss genau diese Preislogik für
sein Unternehmen individuell, unabhängig und zeitnah regeln können.

\section{Erstellung der DSL}\label{erstellungDSL}
Begonnen hat die Erstellung der DSL mit der Vorstellung, dass es im PMS ein
Textbereich gibt in dem die der Text eingefügt und editiert werden kann.
Der Texteditor sollte mindestens ein ``Rich-Text-Editor'' sein, damit der
Domänenexpoerte den Text formatieren kann.
Unter dem Textbereich ist es notwenig, zwei Buttons bereitzustellen. Einen um
die DSL Live anzuwenden und einen um die DSL zu simulieren also zu testen.

Da das zukünftige PMS in Groovy und Grails erstellt wird und Groovy viele
Möglichkeiten der Metaprogrammierung und der DSL erstellung explizit hat, ist es
naheliegend, das Experiment in einer internen DSL umzusetzen. 

Aus der Zieldefinition geht hervor, dass das Resultat der Preisberechnungslogik
eine Tabelle sein muss, die für jeden Tag und jeden Zimmertyp eine Gleitkommazahl als
Preis beinhaltet (Tabelle \ref{zielDefinition}).  

\begin{table}[htpb]
  \centering
  \begin{tabular}{| l | c | r |} 
 	\hline 
  	Datum & Zimmerkategorie & Tagespreis \\
  	\hline 
	1.1.2013 &  Zimmerkategorie1 &  95.00 \\
	1.1.2013 &  Zimmerkategorie2 &  105.00 \\
	2.1.2013 &  Zimmerkategorie1 &  95.00 \\
	2.1.2013 &  Zimmerkategorie2 &  105.00 \\
	3.1.2013 &  Zimmerkategorie1 &  95.00 \\
	3.1.2013 &  Zimmerkategorie2 &  105.00 \\
	\hline 
  	\end{tabular}  
 	\caption{Zielstruktur}
  \label{zielDefinition}
\end{table}	 

Perspektivisch war der Gedanke, dass man von einer Menge alle Elemente dazu
benutzen muss, um den Preis zu bilden. Damit ist gemeint, dass sich die Tage in
einer Menge befinden und auch die Zimmerkategorien Mengenbasierend sind.
Weiterhin besteht eine Berechnungslogik zum größten Teil aus mathematischen
Ausdrücken bzw. Formeln.
Bemerkenswert ist, das die Erstellung der Domänenlogik nicht zuerst auf der
Grundlage des semantischen Modells erstellt wurde sondern rein Intuitiv auf
Basis von bekanntem Domänenwissen. Da sich das semantische Modell als das einer
Bash-Script-Sequenz mit Schleifen herausstellte, ist es nicht verwunderlich,
dass ein Programmierer mit langjähriger Erfahrung das auch ohne Schema erstellen
konnte.
Begonnen wurde mit einem TestTreiber, der eine Textdatei einliest und diese
Interpretiert bzw. evaluiert.
Cliff James hatt das in einem
Tutorial\footnote{http://www.nextinstruction.com/blog/2012/01/08/creating-dsls-with-groovy/}
bewerkstelligt und folgenden Trick angewendet:
Die DSL befindet sich innerhalb einer separaten Datei und ist nach Ausführung
des Einlesecodes in einen interpolierten String umgewandelt. Anschließend wird
dieser String in einen Closure-Block eingefügt.
Da dieser eingefügte String, innerhlab einer ``run'' Methode liegt ist der
Aufruf immer der selbe.
Die DSL wird letztendlich von der Groovy Shell Instanz evaluiert (\cite[S.
368]{koenig2007groovy}). 
Doch ohne den Kontex um die DSL, speziell ausenstehende Werte ist die DSL nutzlos. Daher wird
eine Instanz von Binding (\cite[S. 368]{koenig2007groovy}) dazu benutzt um
Variablen an das Script zu übergeben.
die Binding Instanz wird dazu benutz um die Variable run eine Closure zuzuweisen
die die loadDSL Methode im runner aufruft. Die Binding instanz wird dann an
die GroovyShell instanz übergeben um die Assoziationen zu gewährleisten (Code
Listing \ref{dslrunner.groovy}) .
Mit Groovy-Metaprogrammierung ist es möglich den Kontext einer Instanz zu
wechseln, also die Instanz einer Klasse. Delegate wechselt also zu ``this'' und
damit ist dann die DSL Bestandteil des DSLRunners. Das bedeutet, das alles was
in der DSL-Datei geschrieben wurde jetzt Methoden und Variablen der
DSLRunner-Klasse referenzieren kann.
 
\lstinputlisting[caption={DSL-Runner},
label={dslrunner.groovy},style=default]{code/dslrunner.groovy}

Da nun jedmögliche Textdatei an den DSLRunner übergeben werden konnte, um
Groovy-Script-Code auszuführen ist es dementsprechend auch möglich den Inhalt
des besagten Textfeldes als Input zu benutzen. Diese triviale Implementierung
wurde übersprungen. 
 Eine Schleifeniteration besteht aus einer List oder
einem Abschnitt (Range) gefolgt von der each Methode und der auszuführenden
Closure als Parameter für diese ``each'' Methode \ref{normalGroovyLoop.goovy}.

\lstinputlisting[caption={Orgniale Groovy Schleifenbeispiel
http://groovy.codehaus.org/Collections}, label={normalGroovyLoop.goovy},style=default]{code/normalGroovyLoop.goovy}

Die Versionkontrollhistorie zeigt, dass der erste Eintrag in der DSL aus einer
Schleife über einem Zeitraum von zwei Jahren erstellt wurde. Denn eine finale
Liste kann nur durch eine Schleifenähnliche Funktion erstellt werden
\ref{firstDSLEntry.groovy}.

\lstinputlisting[caption={erster DSL Enturf},
label={firstDSLEntry.groovy},style=default]{code/firstDSLEntry.groovy}

Durch die im theoretischen Teil vorgestellten Kategorien war es nun möglich
anstatt der speziellen Notation für zwei Jahre, bzw. die Instanziierung einer
Dauer (Duration) die Notation 2.years zu verwenden. 

Weiterhin wurde die Instanz
der \texttt{ExpandoMetaClass} der Date-Klasse (Date.metaClass) dazu verwendet,
um eine Methode namens ``bis'' für die Date-Klasse zu definieren, die wieder ein Date
Objekt als Argument entgegennimmt und daraus einen (Zeit)Abschnitt (engl.
Range, Notation: ``start..stop''  ) daraus ableitet.

Durch das Binding Objekt konnte die vordefinierte instanz (new Date()) mit dem
Variablennamen ``heute'' übergeben werden. Diese Variable konnte somit in der
DSL also als solche verwendet werden. 

Die erste Spalte der Zieltabelle ist somit darstellbar, aber die lesbarkeit war
erhelblich durch Sonderzeichen beeinträchtigt. Ziel war nun die Lesbarkeit
erheblich zu steigern indem Sonderzeichen weitestgehend eliminiert werden und
englische Begriffe durch deutsche zu ersetzen.
Zuerst wurde das wort ``each'' durch ``alle'' ersetzt. Das gelang dadurch, dass
die die Bedeutung an sich ``jedes Element'' in einer bestimmten Menge
(engl. Collection) ihren Ursprung hat. Hier wurde wiederrum das
ExpandoMetaObject dafür Benutzt um der Überklasse ``java.util.Collection'' eine
Closure für die neu definierte Eigenschaft ``alle'' zu übergeben. Die Closure
sollte nun eine Iteration über alle Elemente in der Menge leisten und dabei
nochmals eine Closure entgegennehmen in der dann die Operation auf das Element
definiert wird. Ausserdem muss der Deligierte wieder auf die Mengeninstanz
gewechselt werden. Abbilung \ref{alle.groovy} zeigt den Codeabschnitt im
DSLRunner.

\lstinputlisting[caption={Definition von [1,2].alle(\{..\})},
label={alle.groovy},style=default]{code/alle.groovy}

Weiterhin wurde aus 2.years eine neue Kategorie definiert, die die deutsche
Bezeichnung von Jahren benutzt. Also 2.jarhe oder 1.jahr.
Dazu wurde das metaClass ExpandoMetaObject von der Klasse Number
dahingehend verändert, dass solche Konstruktionen möglich werden (Code Listing
 \ref{jahre.groovy}).

\lstinputlisting[caption={Expando Metaclass Jahre},
label={jahre.groovy},style=default]{code/jahre.groovy}

Durch die vorgestellten Syntaxeigenschaften ist es möglich die Klammern
wegzulassen und damit den in Code Listing \ref{finalIterationCode.groovy}
dargestellten DSL-Code zu erzeugen.

\lstinputlisting[caption={Iterationsnotation auf Basis von Kategorien},
label={finalIterationCode.groovy},style=default]{code/finalIterationCode.groovy}

Um diesen gut lesbaren Code noch mehr an die deutsche Ausdrucksweise anzulehnen
ist die Verwendung von Command Expressions hilfreich um eine Fluid DSL zu
erstellen. Das bedeutet, dass in der deutschen Sprache eigentlich folgender
Ausdruck der natürlichste waer: ``alle Tage von heute bis in zwei Jahren
einzeln auflisten und jeden Tag immer als Tag bezeichnen.'' Nun diese ehr lange
Ausdrucksweise ist zwar prezise aber enthält gegenüber einer mit
minimalen Sonderzeichen geschriebenen Notation noch zu viele Begriffe.
Ein valider Kompromiss ist folgender: ``von heute bis 2.jahre alleTage \{ tag
-> \}''.
Dieser Kompromiss wurde ausgehend von der vorhandenen Programmiersprache in der
die DSL ``eingebettet'' sein soll und der subjektiven Empfindungsweise des
Erstellers gemacht. 
Die Präposition ``von'' ist der Name einer Methode, die als Argument ein Datum
akzeptiert und eine Methode als Rückgabewert hat.
``Von'' ist somit eine Methode höherer
Ordnung\footnote{ \url{de.wikipedia.org/wiki/Funktion_höherer_Ordnung} } und
in Code Listing \ref{fluentInterface.groovy} dargestellt.

\lstinputlisting[caption={Fluent Interface Implementierung},
label={fluentInterface.groovy},style=default]{code/fluentInterface.groovy}

Der Rückgabetyp ist eine HashMap mit ``keys'' als Methodennamen und
Closures als ``values'' bzw. dazugehörige ``Methodenkörper''.
Wenn genau das der Fall ist, ist so ein Listeneintrag widerrum ein Objekt an dem
Methoden aufgerufen werden können. Wenn eine Map zurückgegeben wird, dann
identifiziert sich der eintrag der Map anhand des Schlüssels (``bis''). Bis
referenziert somit einen Closurekörper, der ein Argument entgegennimmt das vom
Typ Date ist. Letztendlich gibt diese Closure ein Instanz von Range zurück. Der
Vorteil dabei ist, dass das Argument vom ersten Methodenaufruf (``von(datum)'')
in der Closure des zweiten Methodenaufrufs benutzt werden kann und somit dieses
``Fluent Interface'' eine abgekapselte Einheit darstellt.
Daraus folgt nun folgende Notation neue Notation für die DSL (Code Listing
\ref{finaldateRangeDSL.groovy}).
Durch triviales kopieren der ``alle'' zu ``alleTage'' ExpandoMetaObejkt Instanz
wurde nach dem DRY\footnote{Dave Thomas, interviewed by Bill Venners
(2003-10-10) http://www.artima.com/intv/dry.html} Prinzip die Closure
wiederverwendet und alleTage steht für eine Menge, genauer für eine ObjectRange zur Verfügung.

\lstinputlisting[caption={Fluent Interface Anwendung},
label={finaldateRangeDSL.groovy},style=default]{code/finaldateRangeDSL.groovy}
 
Wie in der Vorbetrachung angemerkt beziehen sich die Preise nicht nur auf den
Zeitraum sondern werden nicht jedem einzelnen Zimmer sondern einer Kategorie
zugeordnet.
Die Information aus der Hostdomäne (Abb. \ref{domainModelDia.png}) bzw. alle
Zimmerkategorien, muss nun in Verbindung mit der DSL gebracht werden.

 
\begin{figure}[h!]
	\begin{center}
	\includegraphics[width=1.3\textwidth,angle=90]{pics/domainModelDia.png}
	\end{center}
	\caption{PMS - Domänen Modell Diagramausschnitt}
	\label{domainModelDia.png}
\end{figure}

Bestenfalls sollte folgende Semantik vollkommen ausreichen:
``Hotel.Zimmerkategorien'' um die Menge abzubilden.
Das Binding Objekt erlaubt nun eine Referenzierung der Domänenmodellinstanzen
mittels dem Grails Framework. Problematisch ist die unterschiedliche Benennung
der DSL komponenten und der Domänenmodelle. Beispielsweise heisst das Hotel im
Domänenmodell ``Estate'' und in der DSL nur ``Hotel''. Wiederrum heissen die
Zimmerkategorien nicht so sondern ``EstateRoomType''. Es ist also notwendig ein
Mapping zu erstellen, dass genau diese Fälle abdeckt. Der Binding Schlüsselwert
für das Estate Objekt ist dann Hotel. 
Da aber die Zimmerkategorien auf kein Feld innerhalb des Domänenmodells
referenziert, muss ein erneutes Mapping erfolgen. Trivial wäre es in dem
Domänenobjekt eine Kopie auf ``estateRoomTypes'' zu machen. Da aber so keine
Kapselung erreichen wird ist es notwendig ein WrapperObjekt zu erstellen und das
an die DSL zu binden (Code Listing \ref{EstateDSLWrapper.groovy}).
 

\lstinputlisting[caption={EstateDSLWrapper.groovy},
label={EstateDSLWrapper.groovy},style=default]{code/EstateDSLWrapper.groovy}

Das Binding ist in Listing \ref{estateBinding.groovy} dargestellt.

\lstinputlisting[caption={estateBinding.groovy},
label={estateBinding.groovy},style=default]{code/estateBinding.groovy}

Analog dazu ist dieses Vorgehen auch mit den definierten Ereignissen
``PriceVariationRange'' durchführbar, welche aus dem Domänenmodell an die DSL
gebunden werden. Da eine Iterationsnotation (``alle'') eingeführt wurde ist es
insgesamt nun möglich {\bf Schleife}n zu schachteln Code Listing
\ref{multipleLoops.dsl}.
 
\lstinputlisting[caption={multipleLoops.dsl},
label={multipleLoops.dsl},style=default]{code/multipleLoops.dsl}

In Code Listing \ref{multipleLoops.dsl} ist Zusätzlich auch schon der erste
{\bf Ausdruck} in Zeile 4 dargestellt. Trivialer weise handelt es sich um
eine Variablendefinition inklusive {\bf Zuweisung}. Dieser Greift auf
die Iterationsvariable ``ZimmerTyp'' zu und referenziert die in dem Wrapper
festgelegte Eigenschaft Grundpreis. Im DomänenModell Estate heisst diese
Klassenvariable ``racRate''.
Die {\bf Variable} Tagespreis ist letztendlich die die modifiziert
werden soll und anschließend in die Ergebnistabelle dem Tag und der
Kategorie zugewiesen werden soll. Die Endtabelle soll in Form einer Liste
definiert werden um dann mit dem Listenoperator ( << ) diese
zu füllen. Die Listennotation ist trivialerweise folgende: ``listenname = []''.

Bisher wurden alle Informationen beschrieben um eine finale implementierung
durchzuführen. Code Listing \ref{trivialSolution.groovy} zeigt, das die
Zielstellung dahingehen erreicht ist, das eine Liste wie in Tabelle \ref{zielDefinition} durch
die DSL berechnet wird.
 
\lstinputlisting[caption={Triviale Lösung des Problems},
label={trivialSolution.groovy},style=default]{code/trivialSolution.groovy}

Das Resultat kann automatisch und transparent gegenüber dem Domänenexperten
durch ein XML oder JSON Mapping an die ``PartnerChannels'' (Booking.com oder HRS) geschickt werden.
Das zu implementieren ist nicht Bestandteil dieser Arbeit.

Da das Grundgerüst der DSL damit geschaffen ist erfolt nun die Anpassung des
Tagespreises durch {\bf Formel}n und {\bf Bedingung}en.

Da der Domänenexperte höchstwahrscheinlich mit Prozenten arbeiten will sollte
es für denjenigen möglich sein diese Zahlen{\bf funktion} einfach benutzen zu
können. Mit Hilfe von Kategorien ist es möglich das zu bewerkstelligen um
letztendlich folgendes DSL
Wort\footnote{http://de.wikipedia.org/wiki/Wort\_\(Theoretische_Informatik\)}
zu erstellen: ``10 prozent Tagespreis''
die Kategorie dazu lautet ist in Code Listing \ref{EnhancedNumber.groovy}
dargestellt.
 
\lstinputlisting[caption={Kategoriedefinition für Prozent},
label={EnhancedNumber.groovy},style=default]{code/EnhancedNumber.groovy}

Wie in der Vorüberlegung schon angedeutet gibt es z.B. eine Preiserhöhung wenn
ein bestimmtes Ereignis eingetroffen ist. So ein wiederkehrendes
Ereignis ist z.B. ein Wochenende. Wenn also der Domänenexperte sich dazu
entscheidet den Preis am Wochenende um 10\% anzuheben sollte er folgendes in der
DSL schreiben können: ``wochenendaufschlag = wenn tag.wochenende dann 10 prozent
tagesPreis''. Wieder wurde hier die Methode eines Fluent Interfaces benutzt wie
bei der Zeitabschnittbestimmung (von(x).bis(y)). In Listing
\ref{ifelseCase.groovy}
 
\lstinputlisting[caption={DSL - if else Ausdruck},
label={ifelseCase.groovy},style=default]{code/ifelseCase.groovy}

Diese Wenn dann Kombination ist wie an der 0 zu erkennen nur für Formeln
einsetztbar. 
Alle Zusätzlichen Erweiterungen für die Date-Klasse sind in Code Listing
\ref{completeDateMetaClass.groovy} dargestellt. Darunter befindet sich auch die
Erweiterung ``getWochenende bzw. wochenende''.

\lstinputlisting[caption={Erweiterungen für die Date-Klasse},
label={completeDateMetaClass.groovy},style=default]{code/completeDateMetaClass.groovy}
 
Dem Domänenexperten wird nun unterstellt, dass er x += 1 als Summierung für x =
x + 1 erlerenen kann. Letztendlich wäre er nun in der Lage folgenden Ausdruck zu
schreiben: ``TagesPreis += wenn Tag.wochenende dann 10 prozent TagesPreis''.
 
Weiter könnte sich der Hotelbetreiber dazu entscheiden folgende modifikation an
dem Tagespreis durchzuführen: Je nach dem wie das Hotel prozentual ausgelastet
ist, wird der Tagespreis, um diesen prozentualen Anteil von einem drittel des
Grundpreises, erhöht oder verringert.
Wieder durch die Binding-Möglichkeit können weitere vordefinierte Variablen
übergeben werden. Z.B. ``binding.gesamtzimmer =
Estate.estateRoomTypes*.count()'' und weiterhin die Anzahl der freien
Zimmer als Methode (Code Listing \ref{predefinedVars.groovy}) 

\lstinputlisting[caption={Vordefinierte Variablen},
label={predefinedVars.groovy},style=default]{code/predefinedVars.groovy}
 
Abschließend soll hier noch weiter die Möglichkeit vorgestellt werden wie auf
die vorher erwähnten bzw. vordefinierten Ereignisse zugegriffen werden kann um
eine Tagespreismanipulation durchzuführen.
Code Listing \ref{finalDSLComplete.dsl} zeigt eine mögliche Form der DSL in der ca. 90\% der
Konzepte Beispielhaft dargestellt sind.  

\lstinputlisting[caption={DSL Beispiel},
label={finalDSLComplete.dsl},style=default]{code/finalDSLComplete.dsl}

Dabei sei nochmal auf die besondere Konstruktion hingewiesen ``tage von: heute,
bis: ereignis.start''. Das ist eine spezielle Notation in Groovy namens
``named parameters''. ``tage'' ist eine Methode, die zwei Parameter entgegennimmt.
``von'' und ``bis'', die auch so benannt werden müssen. Durch das Entfernen der
klammern ist jetzt die dahinterliegenede Struktur zu erkennen. Die Alternative hinter dem Ausdrück im
Kommentar (Zeile 15) ist anders aufgebaut. ``abstand'' ist nun eine Methode, die
eine Closure als Argument entgegennimmt, in diesem Fall einen Abschnitt
(Range).

\section{Das semantischen Modell}

Im Kapitel \ref{erstellungDSL} wurde beschrieben wie die DSL erstellt wurde.
Dabei wurden einige Begriff in bold markiert um die wichtigsten
Meta-Bestandteile unterschwellig zu verdeutlichen. In Abbilung \ref{mm.png}
ist deren Zusammensetztung bzw. das semantische Modell der Preisberechnung
dargestellt.
Dieses ist sehr stark an das einer Skriptsprache angeleht.
In diesem Modell sind nur Berechnungsbestandteile durch Asdrücke und Schleifen
definiert. In einer Scriptsprache ist darüber hinaus noch mehr möglich.
 
\begin{figure}[h!]
	\begin{center}
	\includegraphics[width=0.99\textwidth]{pics/mm}
	\end{center}
	\caption{Preisberechnung semantisches Modell}
	\label{mm.png}
\end{figure}

Die Erstellung dieses Modells, wurde nach der Erstellung der DSL gefertigt um
daraus z.B. eine bessere Dokumentation zu erstellen, die dem Domänenexperten
hilft die DSL zu erstellen.
  
In Abbildung \ref{mmInstanz.png} ist ein Instanzdiagram, der ersten DSL
Zeilen, um das Semantische Modell zu verifizieren.
Durch die Codezeilen-Eigenschaft eines Berechnungsbestandteils kann die
Abarbeitungssequenz dargestellt werden. Die drei Schleifen werden abgebildet und
einzelen Ausdrücke innerhalb der Schleifen, wie z.B. ``wenn TagInnerhalbEreignis
dann 10 prozent tagesPreis'' oder ``Tagespreis = Zimmertyp.Grundpreis''. Die
grau markierte Instanz soll darstellen, das es sich um ein und das selbe
handelt. 

\begin{figure}[h!]
	\begin{center}
	\includegraphics[width=0.99\textwidth]{pics/mmInstanz}
	\end{center}
	\caption{Instanzdiagram der ersten Codezeilen}
	\label{mmInstanz.png}
\end{figure}
 
\chapter{Auswertung}
 
\subsection{Wahl der DSL Variante}
Bewertung des Probands: Lesbarkeit, intuitivem Verständnis und Flexibilität

\subsection{Beurteilung des Domänenexperten}

\subsection{Beurteilung der Meta-Programmierungstools von Grovvy} 
Einfacher da kein aufwand für gui zu betreiben ist.
 
\chapter{Zusammenfassung und Schlussbetrachtung}

 
 
\bibliographystyle{alphadin}
\clearpage\addcontentsline{toc}{chapter}{\bibname}\bibliography{jabref}
\appendix
\renewcommand{\theequation}{A-\arabic{equation}}

\setcounter{equation}{0}  % reset counter \chapter*{Anhang}  % use *-form to

%\chapter{Anhang}
\addcontentsline{toc}{chapter}{Anhang}

%\section{Anhang A - Screenshots}\label{appendixA}


%\listoffigures
\end{document}
