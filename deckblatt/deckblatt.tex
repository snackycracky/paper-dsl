\titlepage

\begin{center}
\includegraphics[width=5cm]{pics/FH-Logo}\vspace{0.5cm}

\par\end{center}

\noindent \begin{center}
\textsf{\textbf{\Large Fachbereich Informatik und Medien - \\
Wissenschaftliches Arbeiten und Schreiben - Master Inf. Prof. Loose
WS 2011/12}}\\ \textsf{\large }\\
\vspace{1cm}

\par\end{center}

\begin{center}
\textsf{\textbf{\huge Untersuchung des sprachorientierten
Programmierparadigmas Anhand von Metaprogrammierung und einer damit
erstellten DSL für Preispoitik in einem Boardinghaus.\\
}}\textsf{}\\ \textsf{}\\ 

\par\end{center}{\Large \par}

\vspace{2cm}


\noindent \begin{center}
{\huge }\begin{tabular}{rl}
Vorgelegt von: & Nils Petersohn B.Sc. \tabularnewline am: &
29.02.2012.\tabularnewline
\end{tabular}
\par\end{center}{\huge \par}

\vspace{1cm}

\begin{abstract}
 
Sprachorientierte Programmierung ist in dieser Arbeit betrachtet wurden.
Dazu wurde eine DSL mit Hilfe der Groovy-MOP erstellt, um daran das Paradigma
zusammen mit dem Domänenexperten unter folgenden Gesichtspunkten zu bewerten:
Lesbarkeit, intuitivem Verständnis und Flexibilität. 
Der Autor beschreibt intensiv die Erstellung der DSL und die
Metaprogrammierungs-Leistungsträger der gewählten Programmiersprache.

\end{abstract}
 

\noindent \begin{center}
\medskip{}
\begin{tabular}{rl}
\tabularnewline
\end{tabular}
\par\end{center}

\newpage{}

%
\begin{comment}
leere Seite nach dem Titelblatt

dann Aufgabenstellung
\end{comment}


\begin{comment}
%\pagestyle{scrheadings}    %Kopfzeile ein

\noindent \begin{center}
\textsf{\textbf{\large Selbstst\"andigkeitserkl\"arung}}
\par\end{center}{\large \par}

\noindent Hiermit erkl\"are ich, dass ich die vorliegende Arbeit zum
Thema

\smallskip{}


\noindent \begin{center}
\textsf{Medienverwaltung mit Eclipse und Equinox}
\par\end{center}

\smallskip{}


\noindent vollkommen selbstst\"andig verfasst und keine anderen als
die angegebenen Quellen und Hilfsmittel benutzt sowie Zitate kenntlich
gemacht habe. Die Arbeit wurde in dieser oder \"ahnlicher Form noch
keiner anderen Pr\"ufungsbeh\"orde vorgelegt.

\medskip{}


\noindent Brandenburg/Havel, den dd.MM.yyyy

\vspace{1.7cm}


\noindent Unterschrift 

\selectlanguage{ngerman}
\end{comment}
%\newpage{}

%\noindent \begin{center}
%\textsf{\textbf{\large Danksagung}}
%\par\end{center}{\large \par}


%\newpage{}
%\begin{abstract}

%\end{abstract}

%\selectlanguage{english}%
%\begin{abstract}
%This is a second abstract in english.

%\newpage{}
%\end{abstract}